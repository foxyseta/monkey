\documentclass[a4paper]{article}
\usepackage[utf8]{inputenc}
\usepackage[italian]{babel}
\usepackage[T1]{fontenc}
\usepackage{amsmath,amssymb,amsthm}
\usepackage{enumerate}

\usepackage{epigraph}
\renewcommand{\epigraphrule}{0pt}
\renewcommand{\textflush}{flushepinormal}
\setlength{\epigraphwidth}{0.275\textwidth}
\renewenvironment{flushepinormal}{}{\vspace*{-\baselineskip}}

\usepackage{fontspec}
\usepackage{graphicx}
\usepackage{hyperref}
\usepackage{verbatim}

\graphicspath{ {./images/} }

\title{
  {
    \fontspec[ Path = fonts/ ]{Symbola}
    \symbol{"1F17C}\symbol{"1F435}\symbol{"1F17D}\symbol{"1F17A}ey
  } \large \\
  \small Relazione del progetto per l'insegnamento di Algoritmi e strutture di
  dati
}

\author{
  Gaia Clerici (\#971338),
  Stefano Volpe (\#969766)
}

\date{
	Universit\`a di Bologna \\
  \today
}

\begin{document}

\maketitle
\thispagestyle{empty}

\begin{figure}[h]
  \includegraphics[width=0.75\textwidth]{monkey}
  \centering
  \caption{\href{https://unsplash.com/photos/daC7ji1EMHM}{una scimmia
  (foto di Bob Brewer)}}
\end{figure}

\pagebreak

\epigraph{Fa' la brava scimmietta.}{\textit{L'uomo con il cappello giallo}}

\tableofcontents

\section{Specifiche}

\subsection{Il gioco: \emph{m,n,k-game}}

Il gioco \emph{m,n,k-game} è deterministico, a turni, a due giocatori, a somma
zero e con informazione perfetta. In una partita, i due agenti si alternano nel
marcare una cella vuota in una griglia di dimensione \emph{m}\texttimes \emph{n}
con un simbolo del proprio colore. Se un giocatore allinea in orizzontale,
verticale o diagonale almeno \emph{k} simboli, questi vince la partita e il suo
avversario la perde. Se non rimangono più celle vuote sulla griglia, la partita
finisce in pareggio.

\subsection{Il torneo: la classifica dei giocatori}

Ogni volta che, all'interno del torneo, un giocatore conclude una partita, egli
guadagna: 
\begin{itemize}
  \item 3 punti in caso di una vittoria come secondo giocatore, ma non a
    tavolino;
  \item 2 punti in caso di una vittoria come primo giocatore o a tavolino;
  \item 1 punto in caso di un pareggio;
  \item 0 punti in caso di una sconfitta.
\end{itemize}

\noindent 
Per ognuna delle configurazioni previste dal torneo, ciascun giocatore gioca
esattamente quattro partite contro ogni altro partecipante, di cui due come
primo giocatore e due come secondo giocatore.

\begin{table}[h!]
\centering
\begin{tabular}{ | c | c | c | }
  \hline
  M & N & K \\
  \hline
  3 & 3 & 3 \\
  \hline
  4 & 3 & 3 \\
  \hline
  4 & 4 & 3 \\
  \hline
  4 & 4 & 4 \\
  \hline
  5 & 4 & 4 \\
  \hline
  5 & 5 & 4 \\
  \hline
  5 & 5 & 5 \\
  \hline
  6 & 4 & 4 \\
  \hline
  6 & 5 & 4 \\
  \hline
  6 & 6 & 4 \\
  \hline
  6 & 6 & 5 \\
  \hline
  6 & 6 & 6 \\
  \hline
  7 & 4 & 4 \\
  \hline
  7 & 5 & 4 \\
  \hline
  7 & 6 & 4 \\
  \hline
  7 & 7 & 4 \\
  \hline
  7 & 5 & 5 \\
  \hline
  7 & 6 & 5 \\
  \hline
  7 & 7 & 5 \\
  \hline
  7 & 7 & 6 \\
  \hline
  7 & 7 & 7 \\
  \hline
  8 & 8 & 4 \\
  \hline
  10 & 10 & 5 \\
  \hline
  50 & 50 & 10 \\
  \hline
  70 & 70 & 10 \\
  \hline
\end{tabular}
  \caption{configurazioni previste dal torneo.}
  \label{table:1}
\end{table}

\subsection{L'obiettivo: il giocatore}

\subsection{L'interfaccia: il pacchetto \texttt{mnkgame}}

\section{Analisi del problema}

\section{Strumenti}

\section{Scelte progettuali}

\section{Conclusioni}

\section{Ricerche future}

\section{Bibliografia}

\end{document}
